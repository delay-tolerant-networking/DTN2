\subsection{Data Store Server Interface}

A data store (DS) retains information across invocations of DTN2
software; a DS server manages one or more data stores. Clients of the
DS server include the BPA, the DP, and the CLA; the DS server may also
be used by applications (A/M module).

DTN2 runs well on small devices, and when augmenting it to support
external data stores we do not want to make any changes that would put
this at risk. One of the reasons that DTN2 can run on small devices is
that it does not require a data store with rich semantics; its data
model is that of a simple persistent hash table.  When DTN2 stores a
C++ object, it first serializes the object as a string of octets, and
then inserts the object into the (abstract) persistent hash table. The
persistent hash table can be implemented using any number of
underlying storage methods---files, a simple persistent hash table for
octet strings, or a more traditional database.

On the other hand, DTN2 also runs well on large devices, and we wish
to support research that will need a data model that is richer than
that offered by opaque octet strings. Decision plane components will
need access to the fields of the persistently stored objects, hence
our DS interface should provide field-level access.

In addition, we envision two general development models for data
stores and DP components. First, one or more simple DS servers will be
developed independently of DP components and made available for
general use; for some DP components these DS servers will be
sufficient for their needs. In other cases the DP will need more
advanced support from the DS server (e.g. will require that the DS
server be able to perform {\it inference}). Advanced DS servers,
designed to support these DP components, will likely provide a very
rich interface to their clients, using a syntax that we cannot know
{\em a priori}.

We must keep in mind, however, that whatever language the DP uses to
communicate with its DS server, DTN2 must also be able to communicate
with that DS server. DTN2 will do this through the use of an external
DS client stub, and we have as a goal to implement a single stub that
DTN2 can use to connect to any data store. In this way, researchers
can innovate in the DP and the DS without having to modify DTN2, and
can connect their components to any instance of DTN2.

In summary, our requirements for the interface include the following.

\begin{enumerate}
\item The DS interface should provide support for efficient storage of 
key-value pair, for small devices with simple DPs. 
\item The DS interface should provide field-level access to
objects, for deployments where DP components need access to the
fields of stored objects, but do not require advanced DS server
support. 
\item The DS interface should provide support for advanced
DS servers (e.g. knowledge bases) that may use data
definition and manipulation languages that are not known to us beforehand.
\item All DS servers should support a simple, common interface
that DTN2 will use. 
\end{enumerate}

Given requirement 4, we have that all DS servers should
support a common interface. What should that interface be? Requirement
1 points us to a simple persistent table of (key, value) pairs, but that
does not give us field-level access (requirement 2). And any simple
common interface will restrict the expressiveness of advanced data
stores / knowledge bases (requirement 3).

Our solution is to provide three levels of functionality:

\begin{itemize}
\item {\tt pair} storage, which DTN2 will use when the
DS server does not support field-based access (e.g. when running on
small devices with a limited DP functionality).
\item {\tt field} storage, which DTN2 will use when the DS
server supports elements with multiple fields (e.g. when running
with a more powerful DP). 
\item An  {\tt advanced} interface used as a general escape
mechanism for DP components, allowing them to communicate with DS
servers using a mutually-agreed-to language (e.g. Prolog, SPARQL,
RDF/OWL, KIF). The advanced interface will not impose any
restrictions on the syntax of the messages passed between the DP and
DS server.
\end{itemize}

In addition, each DS server will support a simple meta-interface that
can be used to learn about its capabilities.

We envision the following usage scenarios:

\begin{itemize}
\item {\em {\tt pair} data store server:} DTN2 stores key-value pairs,
with the data holding an object serialized as an opaque string of octets.
As the values are opaque, DP components must be able
to work without information about the contents of the stored objects.
\item {\em {\tt field} data store server:} DTN2 stores data as objects
with multiple named fields. Decision plane components can retrieve objects from the
DS and inspect the fields of the objects. 
\item {\em {\tt advanced} data store server:} DTN2 and clients other
than DP components treat such a server as they would treat
a simple data store server. Advanced DP components can query
the DS server to learn what advanced data definition and data
manipulation languages are supported by the latter, and
then use these languages to perform advanced operations. As an
example, a DP component might determine that the DS 
server supports SPARQL; the component could then
send SPARQL queries to the DS server.
\end{itemize}

In short, we see three types of DS servers: simple (key-value) {\tt
pair} storage servers; {\tt field} servers; and {\tt advanced} servers
(which will also support {\tt field}-based access).

\subsubsection{Implementation}\label{sec:ds-iccp-impl}

The DS interface is provided using the ICCP (Section~\ref{sec:iccp}), which 
builds upon the external router interface protocol developed by MITRE. We 
require some extensions (as described below) to the current MITRE 
implementation in order to  support the DS interface.

Messages are encoded as XML and transmitted via TCP.  Each
transmission is preceded by a zero-padded, eight-octet, printable
ASCII length argument, which specifies specifying the number of octets of
XML data to follow (e.g. the characters ``00000321'' would precede 321
bytes of XML data).

XML schema (XSD) files for the client-to-DS server and DS
server-to-client interfaces will be provided.

The protocol is asynchronous. Clients can layer a synchronous
interface on top if they so desire (the better to work with DTN2). The
{\em cookie} argument, present in all request messages, can be used to
match reply messages with their corresponding requests. The cookie
argument is not interpreted by the DS server---it is copied directly from
a request message to the corresponding reply, and is entirely for the
use of the client.

All DS servers support the standard storage interface, but simple {\tt
pair} servers can only handle tables with two fields (key and
value). Advanced servers support the full storage interface.

Each data store can hold a number of named tables---collections of
(abstractly) homogeneous objects. The elements in each table have some
number of fields.  As stated above, tables in a {\tt pair}
store are limited to two fields (key and value); tables in other data
stores can have more fields. Each table has a distinguished field that
is used as the key; keys are unique across all elements of the table.

The number of fields per table is fixed at table creation.\footnote{Note that
we may want to change this, e.g., if the DP wants to add arbitrary
attribute/value pairs to stored data.} 

\subsubsection{Parameter Types}

\begin{tabular}{|r|p{5in}|}
\hline
Cookie & string sent by client, returned by server on corresponding response \\ \hline
Data Store Type & {\tt pair}, {\tt field}, {\tt advanced} \\ \hline
DS Handle & an uninterpreted string representing a client's active connection to a data store \\ \hline
Error code & unsigned int returned from data store (details TBD) \\ \hline
Key, Data & byte strings \\ \hline
Keys & a list of encryption keys \\ \hline
Language & a string identifying a language that can be used to communicate with the DS \\ \hline
Name & a character string, used to identify data store names, table names, and field names \\ \hline
Password & a password string \\ \hline
Quota & a 32-bit signed integer representing a storage quota, in MB \\ \hline
User & a string identifying a user \\ \hline
\end{tabular}

\subsubsection{Client to Data Store Server Request Messages}

This section enumerates the request messages that can be sent from
clients (DTN2, DP components, etc.) to the DS
server. The following section enumerates the corresponding reply
messages, which are sent from the DS server to its clients. 
For each message defined in this section named {\em SomeMessage}
there will be a corresponding {\em SomeMessageReply} defined below.

\begin{table}
\centering
\begin{tabular}{|l|c|c|c|c|}
\hline
{\em Store Type} & {\em Data Store} & {\em Table} & {\em Element} & {\em Advanced} \\ \hline \hline
{\tt pair} & ALL & ALL & Put, Get, Del & --- \\ \hline
{\tt field} & ALL & ALL & ALL & --- \\ \hline
{\tt advanced} & ALL & ALL & ALL & ALL \\ \hline
\end{tabular}
\caption{\label{table:supported-messages} Messages supported by each data store type.}
\end{table}

\paragraph {}
{\bf Data store server messages}

Messages sent from the client to operate on the DS server
itself---query its capabilities, create a data store, open, close, and
delete a data store, 

\method{DataStoreCapabilities(cookie)}
{
\metP
    {\em cookie}: character string, returned with reply

\metD
    Request information about the capabilities of the DS server.
    Returns the supported languages, store type, and whether or not
    the DS server supports {\em triggers} (detailed description in
    Section \ref{sec:dsadvmsg}).
}


\method{DataStoreCreate(name, clear, quota, user, password, keys, cookie)}
{
\metP
    {\em name}: name of data store to create\\
    {\em clear}: if true, and data store exists, clear it out\\
    {\em quota (opt)}: maximum size in MB of the data store\\
    {\em cookie}: character string, returned with reply

    Placeholders for authentication:\\
    {\em user (opt)}: User name\\
    {\em password (opt)}: Password\\
    {\em keys (opt)}: Keys for accessing the data store

\metD
    Create the named data store, or, if it exists and the {\em clear}
    flag is set, clearing it.  Authentication parameters are listed
    here as a placeholder; the details of the authentication procedures
    are to be worked out.

}

\method{DataStoreDelete(name, user, password, keys, cookie)}
{
\metP
    {\em name}: name of data store to create\\
    {\em cookie}: character string, returned with reply

    Placeholders for authentication:\\
    {\em user (opt)}: User name\\
    {\em password (opt)}: Password\\
    {\em keys (opt)}: Keys for accessing the data store

\metD
    Delete the named data store.
}

\method{DataStoreOpen(name, lease, user, password, keys, cookie)}
{
\metP
    {\em name}: name of data store to open\\
    {\em lease (opt)}: lease time, in seconds\\
    {\em cookie}: character string, returned with reply

    Placeholders for authentication:\\
    {\em user (opt)}: User name\\
    {\em password (opt)}: Password\\
    {\em keys (opt)}: Keys for accessing the data store


\metD 

Return a handle for the data store. The handle will be valid for the period of
the lease, or until the connection drops, or the client or server is restarted. 
}

\method{DataStoreStat(handle, cookie)}
{
\metP
    {\em handle}: A handle to the data store\\
    {\em cookie}: character string, returned with reply

\metD
    Return a description of the tables in the data store. For now this
    is just a list of the table names.
}


\method{DataStoreClose(handle, cookie)}
{
\metP
    {\em handle}: handle of data store to close.\\
    {\em cookie}: character string, returned with reply

\metD
    Drop the connection to the data store. The handle is invalidated.
}

\paragraph{}
{\bf Table messages}

Each data store can hold multiple named tables. Table messages are used to
create, delete, and obtain information about the tables in a data store.

\method{TableCreate(handle, name, keyname, keytype, list(pair(fieldname, fieldtype)), cookie)}
{
\metP
    {\em handle}: handle of the data store\\
    {\em name}: table name\\
    {\em keyname}: name of key field\\
    {\em keytype}: type of key field\\
    {\em fieldname}: name of a field\\
    {\em fieldtype}: type of the field\\
    {\em cookie}: character string, returned with reply

\metD
    Create a named table, with the specified fields. The name of the
    key field is called out.  The key field should also appear
    in the list of fieldnames and types. TableCreate is used by all
    stores, but only two fields (one named key, and one other) can be
    created when using a {\tt pair} data store. 

    We envision supporting a set of standard simple data types, e.g.
    integers, strings, booleans, etc. Details are TBD.
    
    If there is already a table with that name, the data store returns
    failure.
}

\method{TableDel(handle, tablename, cookie)}
{
\metP
    {\em handle}: handle of the data store\\
    {\em tablename}: table name\\
    {\em cookie}: character string, returned with reply

\metD
    Delete the named table, and all of its data
}

\method{TableStat(handle, tablename, cookie)}
{
\metP
    {\em handle}: handle of the data store\\
    {\em tablename}: table name\\
    {\em cookie}: character string, returned with reply

\metD
    Return information about the named table, including the table's schema
    (the names and types of its fields), the number of elements in the table,
    and, if available, the aggregate size (in MB) of the table.
}

\method{TableKeys(handle, tablename, cookie)}
{
\metP
    {\em handle}: handle of the data store\\
    {\em tablename}: table name\\
    {\em cookie}: character string, returned with reply

\metD
    Return a list of the keys stored in the table.
}

\paragraph{}
{\bf Element messages}

\method{Put(handle, tablename, keyval, list(pair(fieldname, value)), cookie)}
{
\metP
    {\em handle}: handle of the data store\\
    {\em tablename}: table name\\
    {\em key}: key value for element\\
    {\em fieldname}: name of field\\
    {\em value}: value for field\\
    {\em cookie}: character string, returned with reply

\metD
    {\em Put} is used to add an element to a table. Values for each
    field in the table must be specified. 
    If an element with the specified key already exists, the element is replaced.
}

\method{Get(handle, tablename, key, cookie)}
{
\metP
    {\em handle}: handle of the data store\\
    {\em tablename}: table name\\
    {\em key}: key value for element\\
    {\em cookie}: character string, returned with reply

\metD
    {\em Get} is used to retrieve a single element from the table,
    based on key. See {\em Select}, below, for a more powerful query
    interface available with {\tt field} databases.
}

\method{Del(handle, tablename, key, cookie)}
{
\metP
    {\em handle}: handle of the data store\\
    {\em tablename}: table name\\
    {\em key}: key value for element\\
    {\em cookie}: character string, returned with reply

\metD
    Delete the corresponding element from the named table.  
}

\method{Select(handle, tablename, list(pair(fieldname to match, value)), list(fieldname to retrieve), howmany, cookie)}
{ 
\metP
    {\em handle}: handle of the data store\\
    {\em tablename}: table name\\
    {\em fieldname}: name of field\\
    {\em value}: some constant value (integer, string, etc).\\
    {\em howmany (opt)}: maximum number of elements to return\\
    {\em cookie}: character string, returned with reply

\metD
    {\tt Field} and {\tt advanced} data stores provide 
    {\em Select}, a limited query functionality. 
    Fields of each element in a table are compared for equality with
    passed-in constant values.\footnote{I.e. no joins.} If an element
    matches (the fields are equal to the specified constant values)
    the fields of interest of that element are returned.\footnote{This is
    for the case where a table stores objects with many fields, but
    only a few fields are of interest.}

    Returns failure if there are no elements in the table that meet the
    criteria. 

    If the {\em howmany} field is passed it specifies the maximum
    number of elements to return. If zero is passed for {\em howmany},
    returns success (and no elements) if there are any elements that
    meet the criteria.
}

\paragraph {}
{\bf Advanced messages} \label{sec:dsadvmsg}

There are two general-purpose advanced messages for use as an escape
mechanism when working with DS servers that have capabilities
outside the realm of the standard interfaces.  These messages provide
the interface for interacting with a full-fledged knowledge base.

\method{Eval(handle, language, command, cookie)}
{
\metP
    {\em handle}: handle of the data store\\
    {\em language}: the language used by the command\\
    {\em command}: the command itself\\
    {\em cookie}: character string, returned with reply
    
\metD
    The {\em Eval} message is used as a mechanism for clients to send
    commands directly to the DS server, unencumbered by the
    syntax defined above. For example, if the data store is
    implemented using an Oracle database, {\em Eval} can be used to
    send an arbitrary SQL statement. If the data store is implemented
    using Flora-2 (see \cite{Flora2} and \cite{XSB}), {\em Eval} can 
    be used to send a Flora-2 expression.

    If the language is not one of the languages supported by the DS
    server the request will fail.

    Results are returned as an uninterpreted string of octets.  
}

\method{Trigger(handle, language, command, cookie)}
{
\metP
    {\em handle}: handle of the data store\\
    {\em language}: the language used by the command\\
    {\em command}: the command itself\\
    {\em cookie}: character string, returned with reply
    
\metD
    The {\em trigger} message is a generalized form of the standard
    trigger request mechanism found in many data stores and production
    systems. {\em Trigger} sends a command to the DS server,
    just as {\em eval}, but unlike an {\em eval}, which generates only
    a single reply message, a {\em trigger} can generate multiple
    reply messages. 

%    (In fact, {\em trigger} is more of a hint to the system than it is
%    strictly necessary. It is a way for the client to tell the data
%    store to treat the attached command as one that may generate
%    multiple responses, not a single response.)
% Trigger are useful in order to support a callback style or an
% event-condition-action style of operation.  With triggers, the DP 
% need not look at every event of a particular type; rather it can 
% simply wait for a condition to be satisifed within the KB.  Triggers
% therefore allow a mechanism by which the DP can delegate some of the
% processing responsibility to an intelligent data store

    If the data store does not support triggers the request will fail.

    If the language is not one of the languages supported by the DS
    server the request will fail.
}

%%----------------------------------------------------------------

\subsubsection{Data Store Server to Client Reply Messages}

Note: the specifics of error codes are TBD. Assume for now that
possible error codes include SUCCESS and FAILURE, with more likely to
be defined. 

\paragraph {}
{\bf Data store server replies}

\method{DataStoreCapabilitiesReply(cookie, dstype, list(languages),
supports-triggers, error)}
{
\metP
    {\em dstype}: string ({\tt pair}, {\tt field}, or {\tt advanced}).\\
    {\em language}: uninterpreted strings, representing a language
    supported by this DS server (e.g. FLORA-2, KIF,
    RDF/OWL). The meanings of each string is a private contract
    between DS server and DP component authors.\\
    {\em supports-triggers}: boolean, true or false.\\
    {\em cookie}: character string sent on request\\
    {\em error}: error (result) code

\metD
    Information about the capabilities of the DS server.
    The supported languages (strings), store type ({\tt pair}, {\tt
    field}, or {\tt advanced}), and whether or not 
    the DS server supports {\em triggers} (detailed description in
    Section \ref{sec:dsadvmsg}).
}

\method{DataStoreCreateReply(cookie, error)}
{
\metP
    {\em cookie}: character string sent on request\\
    {\em error}: error (result) code

\metD
    Error code indicates whether the data store was successfully
    created or, if creation failed, why.
}


\method{DataStoreDeleteReply(cookie, error)}
{
\metP
    {\em cookie}: character string sent on request\\
    {\em error}: error (result) code

\metD
    Error code indicates whether the data store was successfully
    deleted or, if deletion failed, why (e.g. it was currently in use, 
    it does not exist).
}

\method{DataStoreOpenReply(handle, cookie, error)}
{
\metP
    {\em handle}: handle for opened data store\\
    {\em cookie}: character string sent on request\\
    {\em error}: error (result) code

\metD 

    A handle for the opened data store. The handle will be valid for
    the period of the lease, or until the connection drops, or the
    client or server is restarted.  
}

\method{DataStoreStatReply(list(tablename), cookie, error)}
{
\metP
    {\em tablename}: The name of a table\\
    {\em cookie}: character string sent on request\\
    {\em error}: error (result) code

\metD
    Return the names of the tables in the data store. 
}

\method{DataStoreCloseReply(cookie, error)}
{
\metP
    {\em cookie}: character string sent on request\\
    {\em error}: error (result) code

\metD
    Error code indicates whether data store was closed successfully.
    (One possible reason for failure is that the data store is not
    currently open.)
}

\paragraph{}
{\bf Table replies}

Each data store can hold multiple named tables. Table messages are used to
create, delete, and obtain information about the tables in a data store.

\method{TableCreateReply(cookie, error)}
{
\metP
    {\em cookie}: character string sent on request\\
    {\em error}: error (result) code

\metD
    Error code indicates whether the table could be created. 
}

\method{TableDelReply(cookie, error)}
{
\metP
    {\em cookie}: character string sent on request\\
    {\em error}: error (result) code

\metD
    Error code indicates whether the table could be deleted. 
}

\method{TableStatReply(tablestatus, cookie, error)}
{
\metP
    {\em table-status}: information about the table\\
    {\em cookie}: character string sent on request\\
    {\em error}: error (result) code

\metD
    Specifics of {\em table-status} TBD, but will include information
    about the table's schema (a list containing the names of its
    fields and their types) and the number of elements in the table }

\method{TableKeysReply(list(key), cookie, error)}
{
\metP
    {\em key}: a key from the table\\
    {\em cookie}: character string sent on request\\
    {\em error}: error (result) code

\metD
    The keys for the elements stored in the table. Error code indicates
    success or failure (including no such table).  

}

\paragraph{}
{\bf Element replies}

\method{PutReply(cookie, error)}
{
\metP
    {\em cookie}: character string sent on request\\
    {\em error}: error (result) code

\metD
    Error code indicates whether element was stored.
}

\method{GetReply(list(pair(fieldname, value)), cookie, error)}
{
\metP
    {\em fieldname}: name of field\\
    {\em value}: value for field\\
    {\em cookie}: character string sent on request\\
    {\em error}: error (result) code

\metD
    The requested value. Error code indicates success or failure
    (including no such key, no such table).
}

\method{DelReply(cookie, error)}
{
\metP
    {\em cookie}: character string sent on request\\
    {\em error}: error (result) code

\metD
    Error code indicates success or failure (including no such key, no
    such table).  
}

\method{SelectReply(list(list(pair(fieldname, value))), cookie, error)}
{ 
\metP
    {\em fieldname}: name of field\\
    {\em value}: some constant value (integer, string, etc).\\
    {\em cookie}: character string sent on request\\
    {\em error}: error (result) code

\metD
    Each sub-list consists of (fieldname, value) information from an element
    that met the constraints of the Select. 

    Returns failure if there are no elements in the table that meet the
    criteria, if the table does not exist, if the named fields do not exist.
}

\paragraph {}
{\bf Advanced replies}

\method{EvalReply(result, cookie, error)}
{
\metP
    {\em result}: the result itself, an uninterpreted string of octets\\
    {\em cookie}: character string sent on request\\
    {\em error}: error (result) code
    
\metD
    The response from the DS server for the corresponding {\em
    Eval} message. The payload ({\em result} argument) is whatever the
    DS server sent. The error code is specified by the DS 
    server's Eval message handler.  }

\method{TriggerReply(result, cookie, error)}
{
\metP
    {\em result}: the result itself, uninterpreted string of octets\\
    {\em cookie}: character string sent on request\\
    {\em error}: error (result) code
    
\metD
    A {\em trigger} message is like an {\em Eval}, but may generate
    a second reply messages at an arbitrary time in the future.

    A rigorous definition of {\em trigger} semantics is TBD.
}

