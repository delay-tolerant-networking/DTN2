\subsection{Decision Plane Interface}
\label{sec:dpintf}

This subsection describes the messages between the Decision Plane (DP) and
the Bundle Protocol Agent (BPA). It describes the parameters of each message
and their semantics.

This interface assumes that most of the routing and forwarding related
{\em control} (i.e. decision making) resides in the DP. In
particular, the ``forwarding table'' is in the DP, and therefore there
is no need for route- or next-hop-related information to flow across
this interface.  The DP computes the next hop and issues a command
(request) to the BPA to send a particular bundle to a particular (next
hop) endpoint.

This interface is based on the current DTN2 implementation as implicitly
documented in \cite{MITRE-IF}
and extends that interface, both with new messages and with more parameters
for a message. We have strived to retain the parameter list and the set
of methods as close as possible to current DTN2 implementation.
In each message description, we include an item called ``DTN2 $\Delta$''
($\Delta$ = Difference) which says what the change is relative to
\cite{MITRE-IF}, and refers to the current DTN2 implementation.

We note that this is a ``semantic'' interface, that is, we expect it to
be specifying the kinds of things that go across and their meanings,
not necessarily at a level necessary for implementation. We expect, though,
that an implementation-level specification (e.g., a C++ header file) can 
be generated easily enough from this.

All messages are assumed to be transparently reliably delivered. 

All parameters are assumed to be ``required'' unless specifically
noted to be ``optional'' (Opt).

\subsubsection{Parameter Types}

In order to describe the interface succinctly, we define a few commonly used
complex parameter types.

The main change over~\cite{MITRE-IF} is the introduction of the Global
Bundle or Fragment ID (GBOF-ID), which uniquely identifies a bundle
or fragment globally. The name change is to emphasize two things:
first, that it is a {\em global id} (distinct from any local
sequence-number-based identifiers generated at each node), and that it
identifies either a bundle or a fragment.

\begin{verbatim}

GBOF ID                 := [sourceEID, timestamp, isFragment,
                           fragment-length, frgmtOffset]
Link ID                 := Unique string identifying a link
Link Attributes         := [type, state, peerEID, is-reachable, is-usable,
                           howReliable, howAvailable, reactive-frag-enabled,
                           underlayAddress, clName,
                           all-CLA-specific-link-attributes]
Link Config Parameters  := A set of the following attributes:
                           is-usable, reactive-frag-enabled, underlayAddress,
                           configurable-CLA-specific-link-attributes
ContactAttributes       := [startTime, duration, bps, latency, pktLossProb,
                           LinkAttributes]
custodySignalAttributes := [adminType, adminFlags, succeeded, reason,
                           originalFragOffset, origFragLength,
                           custodySignalSec, custodySignalUsec,
                           originalCreationSec, originalCreationUsec,
                           originalSourceEID]
\end{verbatim}

Parameter types used in current DTN2~\cite{MITRE-IF} are slightly
different in some cases and shown below as a reference for the DTN2
signature given in each message description below.

\begin{verbatim}

bundle                   := [source, dest, custodian, replyto, payload]
linkAttributes           := [name, type, state, nextHop, reliable(yes/no),
                            clName]
linkType                 := [clInfo, linkAttributes]
contactAttributes        := [startTimeSec, startTimeUsec, duration, bps, 
                            latency, link]
custodySignalAtttributes := [adminType, adminFlags, succeeded, reason,
                            originalFragOffset, origFragLength,
                            custodySignalSec, custodySignalUsec,
                            originalCreationSec, originalCreationUsec,
                            originalSourceEID]
\end{verbatim}


\subsubsection{Event Messages}
\label{sec:DP-EventMessages}

The event primitives are triggered by the BPA to notify the DP when a 
reportable event occurs.\\[1em]


\method{EventBundleReceived(GBOF-ID, DestEID, ExpiryTime, Custodian, ReplyTo, BytesReceived, [PreviousHopEID])}
{
\metP
    {\em GBOF ID}: Identifies the bundle or fragment that was received.\\
    {\em Destination EID}: The destination for the bundle.\\
    {\em Expiry time}: Time (using common local clock) at which the
    bundle will expire.\\
    {\em Custodian}: EID of the current custodian of the bundle.\\
    {\em ReplyTo}: EID same as the reply-to field in the bundle header.\\
    {\em Bytes received}: Total number of bytes in the bundle payload, 
    including headers.\\
    {\em Previous Hop EID (Opt)}: The node from which this bundle was most 
    recently sent.\\


\metD
    This is sent by the BPA to the DP whenever a bundle is received.
    The bundle payload itself is retained by the BPA (perhaps in persistent
    store) and the above parameters are sent. We note that the parameters
    constitute only a subset of the information available about the received
    bundle. In the case that the DP requires bundle information not contained
    in the event, it is expected to use the Query/Report message interface
    (refer to section~\ref{sec:DP-QueryReportMessages}).

    The expected action from the DP is to decide whether to re-send
    the bundle and if so how. If it needs to be sent, a sendBundle
    request is made sometime in the future referring to the GBOF-ID
    received in the EventBundleReceived message. The Destination EID is
    used to determine a suitable ``next-hop''. The expiry
    time may be used to prioritize amongst pending bundles
    (e.g. sending soon-to-expire bundles first).

    The previous hop EID is optionally indicated to help DP processing, for
    instance, in flooding-like schemes. It is optional and not mandatory 
    because it might not always be available (e.g. a bundle that arrives on
    a USB stick with no previous hop block).

\metM
    DTN2 Params: (bundle, source, bytesReceived).\\ 
    Remarks: The specified parameters are a strict superset of the DTN2
    parameters. The additional parameters
    (expiry time, previous hop) will help diversify the class of
    routing algorithms possible.

\metR
    EventBundleTransmitted \\
    RequestSendBundle
}

\method{EventDataTransmitted(GBOF-ID, LinkID, TotalBytesSent, TotalBytesSentReliably)}
{
\metP
    {\em GBOF ID}: Identifies the bundle or fragment that was transmitted.\\
    {\em Link ID}: Identifies the link over which it was transmitted.\\
    {\em Total bytes sent}: Number of bytes of data (including headers)
    transmitted.\\
    {\em Total bytes sent reliably}: Number of bytes of data (including
    headers) sent reliably. This could be different from total bytes sent
    if there were no ACKs received for some part of the data.

\metD
    [Note: The event name has been changed from EventBundleTransmitted
    to better reflect the semantics]
    This is generated by the BPA whenever a bundle or a part of a
    bundle is transmitted successfully. The GBOF ID identifies the
    bundle or fragment sent (and if it was fragmented, the subsequent
    fragments currently get different GBOF IDs).
    If the ``total bytes sent'' is less than the payload
    plus header size of the bundle, the DP infers that the bundle has
    been reactively fragmented. The DP may determine whether or not the
    whole bundle has been sent as follows: the source EID and timestamp
    uniquely identify the original bundle, and each fragment identifies
    the range of this bundle that it covers (via the offset/length). With
    this, the DP can match a set of fragments to an original bundle.
     However, not all of the bytes need have been sent
    {\em reliably} for this event. The number of bytes sent unreliably
    is equal to ``total bytes sent'' minus ``total bytes sent
    reliably'' and is assumed to be at the tail (suffix) of the
    bundle/fragment.  No action is expected from DP apart from state
    changes regarding the disposition of the identified bundle, and
    perhaps resending the bundle if not all bytes were sent reliably.
    
\metM
    DTN2 Params: (bundle, contact, bytesSent, bytesReliablySent)\\
    Remarks: The main difference is the use of Link ID rather than contact, 
    but the former is a superset of the latter and this only generalizes the
    parameter. Further, the DP typically does not need to know the contact,
    since it already knows it from the Link ID. Also, the name of the
    event has been changed.\footnote{New name due to M. Demmer, in feedback
    to specification.}
    
\metR
    EventBundleReceived\\
    RequestSendBundle\\
    RequestInjectBundle
}

%\method{EventBundleTransmitFailed(GBOF-ID, LinkID, FailureReason)}
%{
%\metP
%    {\em GBOF-ID}: Identifies the bundle whose transmission failed.\\
%    {\em Link ID}: Identifies the link on which transmission failed.\\
%    {\em Failure Reason}: Code indicating reason for failure.
%
%\metD
%    This event is generated when the BPA is asked to send a bundle
%    (for instance, by a call to RequestSendBundle), and is unable to send
%    any part of the bundle.  Typically (and in the current version of
%    the code), this happens when reactive fragmentation is turned off
%    and the contact goes down.  It also happens (in the current
%    version) when a link is closed---this event is generated for each
%    bundle that is queued on the link that was closed. Upon receipt of
%    this event the DP will typically resend the bundle.
%
%\metM
%    DTN2 Params: (bundle, contact)\\
%    Remarks: Our parameters are a strict superset of the DTN2  parameters. 
%
%
%\metR
%    EventBundleTransmitted\\
%    RequestSendBundle\\
%    RequestInjectBundle
%}

\method{EventBundleDelivered(GBOF-ID, SourceEID)}
{
\metP
    {\em GBOF ID}: Identifies the bundle or fragment delivered.\\
    {\em Source EID}: Identifies the originator of the bundle.

\metD
    This event is generated when the node is a member of the destination 
    endpoint. The DP typically uses this event to note that this is 
    the ``final'' destination and hence no forwarding is required. It may
    also use this event to perform purging of duplicate bundles in the network.

\metM
    DTN2 Params: (bundle, source)\\
    Remarks: No noteworthy difference.

\metR
    EventBundleReceived\\
    EventBundleTransmitted
}


\method{EventBundleExpired(GBOF-ID)}
{
\metP
    {\em GBOF ID}: Identifies the bundle whose lifetime has expired.

\metD
    This event is generated when the lifetime expiration of the identified
    bundle occurs. It is assumed that the bundle is automatically
    deleted by the BPA. The DP is not expected to have any action, although
    it will probably need to clean up state regarding the bundle.

\metM
    DTN2 Parameters: (bundle)\\
    Remarks: No noteworthy difference.

\metR
    None
}

\method{EventBundleSendCancelled(GBOF-ID, LinkID)}
{
\metP
    {\em GBOF ID}: Identifies the bundle whose sending was canceled.\\
    {\em Link ID}: Identifies the link for which the bundle send was
    canceled.

\metD
    This event is sent when a bundle that was previously requested to be 
    sent using RequestSendBundle is canceled upon a subsequent request of 
    RequestCancelBundleSend. The DP uses this as a positive confirmation of its
    request

\metM
    DTN2 does not have this event.

\metR
    RequestSendBundle\\
    RequestCancelBundleSend
}


\method{EventCustodySignal(GBOF-ID, CustodySignalAttributes)}
{
\metP
    {\em GBOF ID}: Identifies the bundle or fragment for which a custody signal
    has been sent.\\
    {\em CustodySignalAttributes}: This is a complex type defined earlier and  
    contains the fragment info, creation and signal timestamps, and whether
    or not custody transfer succeeded and if not the reason.

\metD
    A custody signal indicates successful or unsuccessful transfer of
    custody~\cite{BP-ID}. This event is generated at the current custodian
    when it receives a custody signal from another endpoint. An unsuccessful
    custody transfer may prompt the DP to look for a different next hop,
    based on the reason code.

\metM
    DTN2 Params: (CustodySignalAttributes)\\
    Remarks: Our signature is similar but pulls out GBOF ID redundantly
    for uniformity and ease of coding the DP.

\metR
    EventCustodyTimeout
}


\method{EventCustodyTimeout(GBOF-ID)}
{
\metP
    {\em GBOF ID}: Identifier of the bundle or fragment whose custody has
    timed out.

\metD
    This event is generated when no custody signal was received within
    a configured time. The DP might use this to forward the bundle to
    a different endpoint.

\metM
    DTN2 Params: (bundle, linkType)\\
    Remarks: The DTN2 interface has an extra parameter, namely linkType. 
    The link parameter records the adjacent link on which a bundle was
    originally sent. This information is meant for the bundle's forwarding
    log, and not really necessary for the DP-BPA interface.\footnote{Based
    on email from J. Bush.}

\metR
    EventCustodySignal

}

Until now, we have looked at event messages related to bundle handling. We now
switch to events regarding link dynamics.

\method{EventLinkCreated(LinkID, LinkAttributes, Reason)}
{
\metP
    {\em Link ID}: Name of  the link that was created. Should be
    used as a handle for all future references to that link.\\
    {\em Link Attributes}: A complex type defined earlier that gives
    more information about the link. This includes the type
    (scheduled/discovered etc.), its state (open, available, unavailable etc),
     the peer EID on the
    other end of the link, the convergence layer name, and optionally 
    an indication of how reliable and how available it is expected to be.
    The complete list is preliminarily same as in~\cite{MITRE-IF}.\\
    {\em Reason}: How the link was created. E.g, user, convergence layer, etc.
    A list of current reason codes with meanings is given in~\cite{MITRE-IF}.

\metD
    This message is the first event to be generated with regards to a
    new link. The new link may be created by the user, by the
    convergence layer or any other means. The BPA names this link and
    gives some of the attributes of this link. The state of the link
    upon creation may be dependent on the type of link. Currently, the
    initial state is AVAILABLE for on-demand, scheduled, and static
    links, and UNAVAILABLE for discovered and opportunistic
    links. Upon receipt of this event, the DP is expected to
    initialize data structures for this link in anticipation of the
    contact becoming available some time in the future, if not
    already.

\metM
    DTN2 Params: (linkType, Reason)\\
    Remarks: After expanding the complex types, there are no noteworthy
    differences between DTN2 and specified parameters here. 

\metR
    EventLinkDeleted\\
    EventLinkAvailable\\
    EventLinkOpened\\
    EventLinkAttributeChange\\
    RequestAddLink
}


\method{EventLinkDeleted(LinkID, LinkAttributes, Reason)}
{
\metP
    {\em Link ID}: Identifies the deleted link.\\
    {\em Link Attributes}: Lists the attributes.\footnote{Not convinced this is
    needed since the link is going away, but kept it in deference to DTN2.}\\
    {\em Reason}: Why the link was deleted. A list of current reason codes 
    with meanings is given in~\cite{MITRE-IF}.

\metD
    This event is generated when a link is deleted. The DP is expected to
    clean up the data structures associated with the link but no other
    action is needed. NOTE: The current code does not handle link deletions.
    This message has been specified in anticipation of the code being able
    to  handle it some time in future. It should be implemented after such
    accommodations.

\metM
    DTN2 Params: (linkType, Reason)\\
    Remarks: After expanding the complex types, there are no noteworthy
    differences between DTN2 and specified parameters here. 

\metR
    EventLinkCreated
}

\method{EventLinkAvailable(LinkID, LinkAttributes, Reason)}
{
\metP
    {\em Link ID}: Identifies the link that has become available.\\
    {\em Link Attributes}: More info about the link.\\
    {\em Reason}: Why the link became available.

\metD
    This event message is sent when a link that was unavailable earlier
    now becomes available for use.  The Link Attributes contains the
    latest value of the link attributes. Upon
    receipt of this event, the DP might 
    try and open the link and/or start using the link in its routing
    computations. This event currently pertains only to on-demand,
    static, and scheduled links.

\metM
    DTN2 Parameters: (linkType, reason)\\
    Remarks: After expanding the complex types, there are no noteworthy
    differences between DTN2 and specified parameters here. 

\metR
    EventLinkCreated\\
    EventLinkUnavailable\\
    RequestOpenLink

}

\method{EventLinkUnavailable(LinkID, LinkAttributes, Reason)}
{
\metP
    {\em Link ID}: Identifies the link that has become unavailable.\\
    {\em Link Attributes}: More info about the link.\\
    {\em Reason}: Why the link became unavailable.

\metD
    This event message is sent when a link that was previously available
    now becomes unavailable. The Link
    Attributes contains the latest value of the link attributes.
    Upon receipt of this event, the DP might mark the link 
    as ``down'' and trigger updates to inform other nodes to not use this
    link.

\metM
    DTN2 Parameters: (linkType, reason)\\
    Remarks: After expanding the complex types, there are no noteworthy
    differences between DTN2 and specified parameters here.

\metR
    RequestCloseLink\\
    EventLinkAvailable
}

\method{EventLinkOpened(LinkID, ContactAttributes)}
{
\metP
    {\em Link ID}: Identifies the link which was just opened.\\
    {\em Contact Attributes}: Gives info about the nature of communications
    with this contact---when it starts and stops, data rate, latency, etc.

\metD
    This event message is sent when a link is opened, either upon request
    by the DP, or the operator (typical for on-demand and scheduled links)
    or by the CLA automatically discovering a peer (for discovered and
    opportunistic links). This event is expected by the DP when it issues
    a RequestSendBundle, and
    gives the DP the green signal to go ahead and send bundles to
    this endpoint. The contact attributes may be used by the DP for
    scheduling/prioritizing purposes.

\metM
    DTN2 Params: (contactAttributes)\\
    Remarks: This is similar to the ContactUpEvent of DTN2. Has been renamed
    as it better reflects the semantics in the context of this interface. 
    Modulo this, after expanding the complex types, there are no noteworthy
    differences between DTN2 and specified parameters.

\metR
    EventLinkClosed\\
    EventLinkDeleted\\
    RequestSendBundle

}

\method{EventLinkClosed(LinkID, ContactAttributes)}
{
\metP
    {\em Link ID}: Identifies the link whose peer endpoint went down.\\
    {\em Contact Attributes}: Gives info about the nature of communications 
    with this contact.\footnote{Not convinced this is
    needed since the contact is going away, but kept it in deference to DTN2.}

\metD
    This event message is sent when the link is closed. This typically
    is done by the CLA when it believes that the contact has been down
    so long that it is not worth keeping the link open. The DP may
    also explicitly close the link (see RequestCloseLink). The
    criteria for closing a link may differ depending upon the link
    type.  The LinkClosed event enables the DP to avoid using this
    endpoint as a transit endpoint for sending bundles, for example.

\metM
    DTN2 Params: (contactAttributes)\\
    Remarks: This is similar to the ContactDownEvent of DTN2. Has
    been renamed as it better reflects the semantics in the context of
    this interface. Modulo this, after expanding the complex types,
    there are no noteworthy differences between DTN2 and specified
    parameters.

\metR
    EventLinkOpened
    RequestCloseLink
}

    
\method{EventLinkAttributeChange(LinkID, LinkAttributes, Reason)}
{
\metP
    {\em Link ID}: Identifies the link one of whose attributes has changed.\\
    {\em Link Attributes}: The latest set of attributes.\\
    {\em Reason}: A reason for the change.

\metD
    The message informs the DP of any change in the link attributes,
    in particular, between creation, availability and contact. For
    example, when a link becomes available, the set of attributes is
    updated, but if the attributes change after that, the Link
    Attribute Change event is generated. Changes in the REACHABILITY
    and USABILITY flags are also indicated using this event--note
    that these flags are part of the LinkAttributes complex type.  The
    DP uses this information to decide whether or not to
    include/exclude this link from routing considerations based on the
    new set of attributes.

\metM
    DTN2 Parameters: DTN2 does not have this message. \\
    Remarks: This message is most useful in (multi-hop) wireless links 
    whose properties may change frequently and this may affect routing.

\metR
    EventLinkCreated\\
    EventLinkAvailable\\
    EventLinkOpened\\
    EventContactAttributeChange
}

\method{EventContactAttributeChange(ContactEID, ContactAttributes, Reason)}
{
\metP
    {\em Contact EID}: Identifies the contact whose attributes have changed.\\
    {\em Contact Attributes}: The latest set of attributes for the contact.\\
    {\em Reason}: A reason for the change.

\metD
    This message informs the DP of any change in the contact attributes,
    in particular, after the contact up event has been generated. For
    example, a contact might go into a very deep fade in a valley and its
    ``bps'' parameter may drop dramatically. This event serves to inform this
    to the DP so that routing decisions may be modified accordingly.

\metM
    DTN2 does not have this message.\\
    Remarks: This message is most useful in (multi-hop) wireless links
    whose properties may change frequently and this may affect routing.

\metR
    EventLinkOpened\\
    EventLinkAttributeChange
}


\subsubsection{Request Messages}
\label{sec:DP-RequestMessages}

Request messages are sent by the DP to the BPA. They are ``actions''
that the DP wants the BPA to perform. Requests do not have an explicit
associated response. Instead, some combination of event messages and
timeouts are used to confirm that the requests have been acted
upon.\\[1em]

\method{RequestOpenLink(LinkID)}
{
\metP
    {\em Link ID}: Identifies the link that the DP wants opened.\\

\metD
    This message is currently meant for on-demand links only. It is
    assumed that the link has been created and a EventLinkCreated
    message was generated by the BPA and received by the DP. The Link
    ID used here is the same as the Link ID given in the
    EventLinkCreated interface.  The DP makes this request when it
    wants to send a bundle to a peer EID on this link, and expects an
    EventLinkAvailable message in response.  The link is kept open
    until a RequestCloseLink is issued until deemed idle (see
    section~\ref{sec:linktypes}).

\metM
    DTN2 Parameters: (link)\\
    Remarks: We have two additional parameters, attributes and duration. 
    This will give more flexibility in how to manage the link. 

\metR
    EventLinkAvailable\\
    RequestCloseLink
}

\method{RequestCloseLink(LinkID)}
{
\metP
    {\em Link ID}: Identifies the link that the DP wants closed.

\metD
    This request closes a link that the DP is no longer interested in
    keeping open. However, the DP cannot assume that the BPA actually closed
    the link. The BPA may not close the link, for example, if other
    DP processes had wanted the link open. If the DP needs confirmation
    (it may not, typically, need it), then it should wait for the
    Link Unavailable event.

\metM
    DTN2 does not have this message.

\metR
    RequestOpenLink\\
    EventLinkUnavailable

}

\method{RequestAddLink(LinkType, PeerEID, CLAName, [LinkConfigParameters])}
{
\metP
    {\em Link Type}: The type of link being added.\\
    {\em Peer EID}: The endpoint ID for the peer of the link being added.\\
    {\em CLA Name}: The CLA the link being added should use.\\
    {\em Link Config Parameters (Opt)}: Describes the link to add. 

\metD
    This request is meant for the case when the DP knows about certain links
    that the CLA does not and would like to have them
    added. The request can supply link attributes. After issuing this,
    the DP waits for a LinkCreated event to be able to use the link. This
    event is the confirmation that the request was satisfied.

\metM
    DTN2 does not have this request.

\metR
    EventLinkCreated
}

\method{RequestDeleteLink(LinkID)}
{
\metP
    {\em Link ID}: Identifies the link to be deleted. Taken from the
    EventLinkCreated call.

\metD
    This request is for deleting links that the DP added using the
    RequestAddLink command. The DP should only delete links that it added,
    and thus, the LinkID should only identify a link that was created
    due to a RequestAddLink call from the DP. The EventLinkDeleted confirms
    that the request went through.

\metM
    DTN2 does not have this request.

\metR
    RequestAddLink \\
    EventLinkCreated \\
    EventLinkDeleted
}
    
\method{RequestReconfigureLink(LinkID, LinkConfigParameters)}
{
\metP
    {\em Link ID}: Identifies the link to be reconfigured.\\
    {\em Link Config Parameters}: The new set of attributes.

\metD
    Link parameters may be changed after creation time with this
    request. Changes are applied immediately where it makes sense, even if the
    link is open. The Link Config Parameters parameter updates link attributes.
    The EventLinkAttributeChange confirms that this request went through.

\metM
    DTN2 does not have this request.

\metR
    RequestAddLink\\
    RequestDeleteLink\\
    EventLinkAttributeChange
}

    
\method{RequestSendBundle(GBOF-ID, LinkID, ForwardAction, [FragmentSize, FragmentOffset, ExtensionBlock])}
{
\metP
    {\em GBOF ID}: Identifies the bundle that the DP wants sent.\\
    {\em Link ID}: Identifies the link over which the DP wants the bundle
    sent.\\
    {\em Forward Action}: One of ``forward to one next hop'', ``forward a
    copy''\\
    {\em Fragment Size (Opt)}: To support proactive fragmentation, this field 
    specifies that only this many bytes of the bundle ought to be sent.
    Optional parameter.\\
    {\em Fragment Offset (Opt)}: To support proactive fragmentation, this 
    field specifies the offset of the size above.\\
    {\em Extension Block (Opt)}: To be attached after the bundle protocol 
    header, contains DP-specific information to be piggybacked to DPs in other
    endpoints. Optional parameter.

\metD
    As discussed earlier, received bundles are retained by the BPA, with
    the event giving some important information to the DP. RequestSendBundle
    message identifies a retained bundle using the GBOF ID that was given
    in the BundleReceived event and asks it to be sent over a link
    identified using a Link ID that was given to it in the Link Created
    event. The Forward Action further qualifies this transmission, in
    particular whether it wants a copy retained for further forwarding
    (as may be done in Epidemic Routing). 

    If the Fragment Size field is specified, then the BPA is expected
    to fragment the specified bundle and send only the fragment. The
    BPA retains the rest of the bundle awaiting further send or
    fragment-and-send instructions from the DP.  The extension block
    is created by the DP and is requested to be attached verbatim
    after the BP header. This facility is useful for protocols that
    require control information to be piggybacked on bundles
    (e.g.~\cite{MaxProp}).

    A BundleTransmitted event following this request serves to confirm
    that the action was indeed performed. Note that there is no event for
    the failure of bundle transmission (such an event was there in a previous
    version and has been removed). Therefore, the DP is responsible for
    using some method to decide whether to resend the bundle. For example,
    it could use a timeout and then cancel the sending of the previous 
    bundle, and if appropriate, resend it on another link. 


\metM
    DTN2 Params: (bundleId, link, forwardAction)\\
    Remarks: We have two extra parameters, both optional---the fragment
    size, to support proactive fragmentation, and the extension 
    block, to support more sophisticated routing protocols.

\metR
    EventBundleTransmitted\\
    RequestCancelSendBundle
}

\method{RequestBroadcastSendBundle(GBOF-ID, ForwardAction, [FragmentSize, FragmentOffset, ExtensionBlock])}
{
\metP
    {\em GBOF ID}: Identifies the bundle that the DP wants sent.\\
    {\em Forward Action}: One of ``forward to one next hop'', ``forward a
    copy''\\
    {\em Fragment Size (Opt)}: To support proactive fragmentation, this field
    specifies that only this many bytes of the bundle ought to be sent.
    Optional parameter.\\
    {\em Fragment Offset (Opt)}: To support proactive fragmentation, this
    field specifies the offset of the size above.\\
    {\em Extension Block (Opt)}: To be attached after the bundle protocol
    header, contains DP-specific information to be piggybacked to DPs in other
    endpoints. Optional parameter.

\metD
    This is relevant for broadcast channels only, not point-to-point. In
    this case, a node might wish to simply transmit the bundle without having
    it be intended for anyone in particular. Such broadcasting is a key part
    of MANET operation, one of the network types envisaged for DTNs. Unlike
    a RequestSendBundle to each and every link (which may be considered a
    broadcast too), the key here is that there is no link specified---indeed,
    the node may not even know which other nodes receive this. Intended uses
    are for bundle flooding, epidemic routing and transmitting DP probes.

    The field semantics are identical to that of RequestSendBundle. 

    A BundleTransmitted event following this request serves to confirm that
    the action was indeed performed.

\metM
    DTN2 does not have this message\\

\metR
    RequestSendBundle\\
    EventBundleTransmitted\\
}

\method{RequestCancelBundleSend(GBOF-ID, LinkID)}
{
\metP
    {\em GBOF-ID}: Identifies the bundle whose sending needs to be canceled.\\
    {\em Link ID}: Identifies the link for which the bundle sending must be
    canceled.

\metD
    After having issued a RequestSendBundle, the DP may wish to retract it,
    if possible. This could happen for instance, if the transmission failed
    and the DP wants to try another bundle, or if it found out that the
    bundle has been delivered and does not need to be sent anymore, or  
    for other reasons. 

\metM
    DTN2 Message: CancelBundleRequest(bundleId, link)\\
    Remarks: We presume the CancelBundleRequest is to cancel the {\em sending}
    of a bundle and not something else, like deleting a bundle. The ``link''
    parameter is not present in our interface for reasons mentioned above.

\metR
    RequestSendBundle\\
    EventBundleSendCancelled
}

\method{RequestInjectBundle(SourceEID, DestEID, payload, [ReplyTo, Custodian, Priority, Expiration, Forward Action])}
{
\metP
    {\em Source EID}: The EID of this DP/router (the one that is
    injecting the bundle).\\
    {\em Dest EID}: Identifies the endpoint, typically a peer, for which 
    the injected bundle is meant.\footnote{Can a bundle be injected for a
    non-peer endpoint?}\\
    {\em Payload}: The information content (data part) of the bundle. Base64
    encoded~\cite{MITRE-IF}\\
    {\em ReplyTo (Opt)}: Where delivery notification must be sent.\\
    {\em Custodian (Opt)}: Current custodian.\\
    {\em Priority (Opt)}: The priority assigned to this bundle.\\
    {\em Expiration (Opt)}: The expiration time of this bundle.\\
    {\em Forw Action (Opt)}: Forwarding action type as in RequestSendBundle.

\metD
    An injected bundle is meant for peer DP(s). We envisage that DP
    will subscribe to a well known endpoint like {\tt dtn://ext.dp}
    and injected bundles will have the destination EID set as {\tt
    peer-EID/ext.dp/foo} where {\tt foo} is a service tag
    (e.g. bbn-algo-4). This message only injects a bundle, does not
    actually cause it to be sent. Sending of an injected bundle should
    be done using the RequestSendBundle command.  An injected bundle
    undergoes pretty much the same kinds of operations as a
    ``regular'' bundle. A EventBundleTransmitted message from the BPA
    will serve to confirm that the injected bundle has been
    transmitted.\\

    The DP can also, optionally, specify the other fields that go into
    the BP header, namely the ReplyTo, Custodian, Priority and
    Expiration.\footnote{This is not mandatory because the control
    protocol using the injected bundle usually has a way to get rid of
    the injected (control) bundles once their utility is
    over. E.g. using sequence numbers, timestamps etc. So there is no
    need to additionally mandate an expiration field. However, for control
    protocols that don't have this facility, this is available as an option.}
     BPA uses defaults
    in much the same way it treats application bundles.

\metM
    DTN2 Params: (source,destination,replyto,custodian,link,fwdAction,priority,expiration,payload)\\
    Remarks: No noteworthy difference.

\metR
    RequestSendBundle\\
    EventBundleTransmitted\\
}

\method{RequestDeleteBundle(GBOF-ID)}
{
\metP
    {\em GBOF-ID}: Identifies the bundle that the DP wants deleted.

\metD
    Recall that our architecture calls for the BPA to retain control of
    the bundle itself (storing etc.) while passing relevant parameters
    to the DP. If the DP determines that a bundle no longer need be
    stored, persistently or otherwise, it issues this command to delete
    the bundle. The DP may want to delete the bundle if it is informed
    through peer-DP mechanisms that a copy of the bundle was delivered.  
    This request is useful for replication oriented forwarding, for
    example, Epidemic Routing. There is no event for confirming that
    the bundle was deleted. 

\metM
    DTN2 does not have this request message. 

\metR
    RequestSendBundle
}

\method{RequestSetCLAParams(Parameters)}
{
\metP
    {\em Parameters}: A list of (key, value) pairs.

\metD
    This is a ``pass-through'' request to the BPA to set the convergence
    layer adapter parameters. The BPA is expected to invoke the message to
    set CLA parameters using the (key, value) pairs in Parameters. The key
    identifies the name of the parameter and the value is what the 
    CLA is supposed to set that parameter to.

\metM
    DTN2 does not have this request. 

\metR
    None
}

\subsubsection{Query/Report Messages}
\label{sec:DP-QueryReportMessages}

Query/Report messages consist of a set of ``paired'' messages---a Query
message from the DP to the BPA requesting certain information from the BPA and
a report message for that query from the BPA to the DP delivering the
requested information, if possible. Unlike the Request-Event messages which
are only loosely tied, there is a tighter coupling between query and report,
in particular a report identifies which query is being responded to.

We have separate query/report messages for bundle and link. As will be
apparent after reading the structure of the messages, there really isn't
a need to separate it, but we have done so in deference to DTN2 which
had it separate.\\[1em]

\method{QueryBundleAttributes(QueryID, QueryParams)}
{
\metP
    {\em  QueryID}: Identifies the query (a handle) to tie the report.\\
    {\em  QueryParams}: A list of keys. The key identifies which information
    object is asked for. A list of key types, mutually agreed upon by DP and
    BPA to be defined somewhere.\footnote{Not currently included in this
    document.}

\metD
    This query is for getting the value of any subset of a number of
    pre-defined bundle attributes, including extension block contents, 
    including location and contents of injected (control)  bundle payload. 
    A unique query id is passed and a
    report message containing the query id is awaited. The report contains
    the requested information (key, value) pairs (see below).

\metM
    DTN2 has a bundleQuery message but its parameters are not defined.

\metR
    ReportBundleAttributes
}

\method{ReportBundleAttributes(QueryID, ReportParams)}
{
\metP
    {\em QueryID}: Identifies which query is being responded to.\\
    {\em ReportParams}: A list of (key, value) pairs.

\metD
    This message is a response to a query and reports the latest value
    of the bundle attributes that were queried for.  The query id ties
    the report to the query. The ReportParams contains (key, value)
    pairs, where each key is taken from the list of keys supplied with
    the corresponding query, and the value is the value of whatever is
    requested for by the key.

\metM
    DTN2 Message: bundleReport(bundleType)\\
    Remarks: The difference is that the specified message is more generalized,
    DTN2 message is a specific key, value pair, one which dumps
    whatever is known about the bundle.

\metR
    QueryBundleAttributes
}

\method{QueryLinkAttributes(QueryID, LinkID, QueryParams)}
{
\metP
    {\em  QueryID}: Identifies the query (a handle) to tie the report.\\
    {\em  Link ID}: Identifies the link whose attributes are sought.\\
    {\em  QueryParams}: A list of keys. The key identifies which information
    object is asked for. A list of key types, mutually agreed upon by DP and
    BPA is defined somewhere.\footnote{Again probably not as part of the arch
    document?}

\metD
    This query is for getting the value of any subset of a number of
    pre-defined link attributes. A unique query id is passed and a
    report message containing the query id is awaited. The report contains
    the requested information (key, value) pairs (see below).

\metM
    DTN2 has a linkQuery but its parameters are not defined.

\metR
    ReportLinkAttributesReport

}

\method{ReportLinkAttributes(QueryID, ReportParams)}
{
\metP
    {\em QueryID}: Identifies which query is being responded to.\\
    {\em ReportParams}: A list of (key, value) pairs.

\metD
    This message is a response to a query and reports the latest value
    of the link attributes that were queried for.  The query id ties
    the report to the query. The ReportParams contains (key, value)
    pairs, where each key is taken from the list of keys supplied with
    the corresponding query, and the value is the value of whatever is
    requested for by the key.

\metM
    DTN2 Message: linkReport(linkType)\\
    Remarks: The difference is that the specified message is more generalized,
    DTN2 message is a specific key, value pair, one which dumps
    whatever is known about the link.

\metR
    QueryLinkAttributes
}
