\subsection{Convergence Layer Adapter Interface}

A Convergence Layer Adapter is responsible for communication with the underlay
networks available to the BPA. The CLA provides transmission and reception of
bundles, and responds to requests to open or close links to neighbor endpoints.
Where possible, the CLA is responsible for performing DTN neighbor discovery
and measuring link characteristics.

The CLA interface described here is based on the DTN2 ConvergenceLayer, with
modifications to create an External CLA interface similar in concept to the
External Decision Plane (Router) interface. Additional functionality is 
designed to support discovery and the DP.

Each CLA provides the functionality for use of a particular kind of underlay
network. Multiple ``interfaces'' may be created by each CLA; each interface
refers to a single underlay network available to the node. Interfaces are
responsible for receiving bundles from peer nodes, and performing neighbor
discovery. Discovery is performed separately on each interface.

A potential or existing connection to another endpoint is represented by a
link. As in DTN2, a given link is associated with a single CLA. If the node
can reach a given endpoint via more than one CLA, then there will be a link
for each. Links will have CLA-specific parameters providing the necessary
information for opening the link, such as the WEP key for 802.11 or the IP
address and port for the TCP and UDP CLAs.

It is possible for a single CLA to have more than one link with the same
destination EID. This would generally only be necessary for situations where
there are different link parameters, and certain bundles should be sent with
one set of parameters or another. An example of such a situation is a TCP
CLA on a system with two hardware interfaces, eth0 and eth1.  If which
hardware interface the bundle is sent from is irrelevant, than only one link
to a given EID is needed and the CLA will use the default hardware
interface. But if the default route for eth1 is a router that tunnels
traffic in IPsec and the default route for eth0 does not tunnel but has much
better bandwidth, and we wanted to be able to tunnel sensitive bundles and
send other bundles on the higher-bandwidth link, two links would be
needed. An additional link parameter, the IP address to send from, would
then constrain each link to eth1 or eth0, so that bundles sent on that link
are transmitted correctly.

Link parameters may constrain how the CLA transmits bundles in ways other
than a hardware interface. For example, the link parameters may require a
CLA to use only a particular 802.11 SSID. Each CLA provides a set of link
parameters specific to it, and can use them to constrain transmission of
bundles, or provide a default where parameters are optional (such as the IP
address to send from).

\subsubsection{Parameter Types}

The CLA interface uses some of the same parameter types as the DP
interface. These include:

\begin{verbatim}

GBOF ID                 := [sourceEID, timestamp, isFragment,
                           fragment-length, frgmtOffset]
Link ID                 := Unique string identifying a link
Link Type               := PERSISTENT, ON_DEMAND, SCHEDULED, PREDICTED,
                           or OPPORTUNISTIC
Link Attributes         := [type, state, peerEID, is-reachable, is-usable,
                           howReliable, howAvailable, reactive-frag-enabled,
                           underlayAddress, clName,
                           all-CLA-specific-link-attributes]
Link Config Parameters  := A set of the following attributes:
                           is-usable, reactive-frag-enabled, underlayAddress,
                           configurable-CLA-specific-link-attributes
ContactAttributes       := [startTime, duration, bps, latency, pktLossProb,
                           LinkAttributes]


\end{verbatim}

Additional parameters include:

\begin{verbatim}

CLA Parameters          := A list of (key, value) pairs
Interface Parameters    := A list of (key, value) pairs
Interface ID            := Unique string identifying an interface
Link State              := OPEN, AVAILABLE, UNAVAILABLE, or BUSY
Bundle Location         := How to get the bundle to send or where an incoming 
                           bundle was placed (from the DS or elsewhere)

\end{verbatim}

Interfaces and links have both parameters and attributes associated with
them. Parameters are used to control behavior, such as whether to perform
fragmentation or what encryption key should be used, while attributes
describe state, such as latency or error rate. Note that link attributes
also include the REACHABLE and UNUSABLE flags.

\subsubsection{Request Messages}

Requests to the CLA are made by the BPA. Some requests may originate from
the DP and are then passed to the CLA by the BPA.\\[1em]

\method{RequestSetCLAParams(CLAParameters)}
{
\metP
    {\em CLA Parameters}: A list of (key, value) pairs.

\metD
    This request sets the CLA parameters. These parameters apply to the
    behavior of the CLA as a whole, not individual interfaces or links. The
    key identifies the name of the parameter and the value is what the CLA is
    supposed to set that parameter to. If any of the parameters is invalid,
    no parameters are set.

    After a CLA has been added (see RequestAddCLA), the BPA should use this
    message to tell the CLA the canonical EID of this node (for use in
    discovery) and where to store incoming bundles. The CLA should store
    bundles as it receives them in the manner specified by this BPA-provided
    Bundle Location so that the BPA can read them upon receiving
    EventBundleReceived messages.

    For basic topology management in the CLA, the CLA should at least support a
    configuration parameter that instructs it to automatically create a link
    and mark it as REACHABLE and AVAILABLE when a new EID is
    discovered. Otherwise new links must be created by RequestAddLink.

    CLAs may support more sophisticated topology management functionality,
    which would be configured/enabled via this request as well. For example,
    if configured, the CLA would delete discovered links that have been
    UNREACHABLE for longer than a certain time, based on the presumption that
    the endpoint will not be reachable again any time soon.

\metR
    EventCLAParamsSet
}

\method{RequestSetInterfaceDefaults(InterfaceParameters)}
{
\metP
    {\em Interface Parameters}: A list of (key, value) pairs. The set of key
    types is CLA-dependent.

\metD
    This message sets default values for interface parameters. These values are
    used when a new interface is created, unless they are overridden at
    creation time by parameters in RequestAddInterface. Changes to the default
    parameters do not change any preexisting interfaces. If any of the
    parameters is invalid, no parameters are set.

\metR
    RequestAddInterface\\
    RequestReconfigureInterface
}
    
\method{RequestSetLinkDefaults(LinkConfigParameters)}
{
\metP
    {\em Link Config Parameters}: A list of link attributes for which to set
    defaults.
%The set of attributes is CLA-dependent.

\metD
    This message sets default values for link parameters. These values are
    used when a new link is created, unless overridden at creation time by
    parameters in RequestAddLink.\footnote{Some parameters may be determined
    by discovery or a measurement process implemented within the CLA.} Changes 
    to the default parameters do not change any
    preexisting links. If any of the parameters is invalid, no parameters are
    set.

\metR
    RequestAddLink\\
    RequestReconfigureLink
}

\method{RequestAddInterface(InterfaceID, [InterfaceParameters])}
{
\metP
    {\em Interface ID}: Identifies the interface to be added.\\
    {\em Interface Parameters (Opt)}: A list of (key, value) pairs. The set
    of key types is CLA-dependent.

\metD
    This requests that the CLA add an interface with the provided name. Any
    default parameters for the interface will be used unless overridden by
    the InterfaceParameters. Bundles sent from another node cannot be
    received, nor can discovery be performed, until the interface is up.
    If any of the parameters is invalid, the interface is not added.

\metR
    EventInterfaceAdded\\
    RequestReconfigureInterface\\
    RequestSetInterfaceDefaults
}

\method{RequestReconfigureInterface(InterfaceID, [InterfaceParameters, UpOrDown, DiscoveryOrNot])}
{
\metP
    {\em Interface ID}: Identifies the interface.\\
    {\em Interface Parameters (Opt)}: A list of (key, value) pairs. The set
    of key types is CLA-dependent.\\
    {\em Up Or Down (Opt)}: This field specifies whether the interface should
    be brought down or up.\\
    {\em Discovery or Not (Opt)}: This field specifies whether or not DTN
    neighbor discovery should be performed on the interface.

\metD
    This request allows the specified interface to be reconfigured. If
    provided, the Interface Parameters are the same as those set in
    RequestSetInterfaceDefaults and RequestAddInterface; the changes are to be
    applied immediately (if it makes sense). Interfaces are brought up or
    down with the Up Or Down field. An interface starts down when added, and
    cannot perform discovery until it is brought up. The Discovery Or Not
    field allows changing whether discovery is enabled; an interface may be
    up without discovery being enabled. If any of the parameters is invalid, no
    parameters are changed.

\metR
    EventInterfaceReconfigured
}

\method{RequestDeleteInterface(InterfaceID)}
{
\metP
    {\em Interface ID}: Identifies the interface to be deleted.

\metD
    This message requests that the interface itself brought down and
    deleted. Existing links that happen to use the same underlay network
    as the interface are not affected. However, depending on how the CLA
    works, such links may no longer be able to send traffic and would
    eventually become UNAVAILABLE (if applicable).

\metR
    None
}

\method{RequestAddLink(LinkID, LinkType, PeerEID, [LinkConfigParameters])}
{
\metP
    {\em Link ID}: Identifies the link to be created.\\
    {\em Link Type}: The type of link being added.\\
    {\em Peer EID}: The endpoint ID for the peer of the link being added.\\
    {\em Link Config Parameters (Opt)}: Describes the link to add.
% The set of attributes is CLA-dependent.

\metD
    This message requests that the CLA add a link. The link is not to be
    opened automatically by the CLA when added.
%\footnote{Unless it's AlwaysOn?} 
The link parameters can include the link's UNUSABLE flag.
%    The underlay address of the peer (called ``nextHop'' in DTN2), if
%    provided, is also a link parameter.
    If any of the parameters is invalid, the link is not added.

\metR
    EventLinkCreated
}

\method{RequestReconfigureLink(LinkID, LinkConfigParameters)}
{
\metP
    {\em Link ID}: Identifies the link to be reconfigured.\\
    {\em Link Config Parameters}: The new set of attributes.
% The set of attributes is CLA-dependent.\\

\metD
    Link parameters may be changed after creation time with this
    request. Changes are applied immediately where it makes sense, even if the
    link is open. This request is also used to set or clear the link's
    UNUSABLE flag. If any of the parameters is invalid, no parameters are
    changed.

\metR
    RequestAddLink
}

\method{RequestDeleteLink(LinkID)}
{
\metP
    {\em Link ID}: Identifies the link to be deleted.

\metD
    This is a request to close the link if it is open and then delete it.

    Note that if the CLA is performing neighbor discovery, and has been
    instructed to create links when peers are found, the CLA may recreate
    this link if it is re-discovered.

\metR
    EventLinkDeleted
}

\method{RequestOpenLink(LinkID)}
{
\metD
    See DP interface (section~\ref{sec:DP-RequestMessages}).

\metR
    RequestCloseLink
}

\method{RequestCloseLink(LinkID)}
{
\metP
    {\em Link ID}: Identifies the link that the DP wants closed.

\metD
    This message requests that the CLA close the link if it is open.

\metR
    RequestOpenLink
}

\method{RequestSendBundle(GBOF-ID, LinkID, BundleLocation)}
{
\metP
    {\em GBOF ID}: Identifies the bundle that should be sent.\\
    {\em Link ID}: Identifies the link over which the bundle should be sent.\\
    {\em Bundle Location}: Specifies from where the CLA should retrieve the
    bundle to be sent.

\metD
    This is a request to send the bundle. There is not a parameter for bundle
    blocks; the assembly of the bundle blocks into a complete bundle must
    be performed by the BPA. The location of this complete bundle is specified
    by the Bundle Location field.

\metR
    EventBundleTransmitted\\
    EventBundleTransmitFailed\\
    RequestCancelBundleSend
}

\method{RequestCancelBundleSend(GBOF-ID, LinkID)}
{
\metP
    {\em GBOF-ID}: Identifies the bundle whose sending needs to be canceled.\\
    {\em Link ID}: Identifies the link for which the bundle sending must be
    canceled.

\metD
    This is a request that the CLA remove the specified bundle from all
    transmission queues if possible. The CLA will post EventBundleSendCancelled
    if the cancellation succeeds. If the CLA is currently transmitting the
    specified bundle, several options may be available. It may still be
    possible for the CLA to cancel the entire bundle transmission, in which
    case the cancellation succeeds. If it is not possible to cancel the
    transmission, the CLA could continue to transmit the bundle, or create a
    reactive fragment. In these cases, the EventBundleTransmitted message would
    be sent as normal.

\metR
    EventBundleSendCancelled\\
    RequestSendBundle
}

\subsubsection{Event Messages}

The CLA sends event messages to the BPA.\\[1em]

\method{EventCLAParamsSet()}
{
\metP
    None

\metD
    This message is sent when a RequestSetCLAParams message has been
    successfully processed.

\metR
    RequestSetCLAParams
}

\method{EventInterfaceAdded(InterfaceID)}
{
\metP
    {\em Interface ID}: Identifies the interface that was added.

\metD
    This event message is sent when an interface has been successfully added
    by the CLA. The interface will not be up until it has been brought up
    by a RequestReconfigureInterface message.

\metR
    RequestAddInterface\\
    RequestReconfigureInterface
}

\method{EventInterfaceReconfigured(InterfaceID)}
{
\metP
    {\em Interface ID}: Identifies the interface that was reconfigured.

\metD
    This event is sent after reconfiguring an interface in response to
    the RequestReconfigureInterface message.

\metR
    RequestReconfigureInterface
}

\method{EventEIDReachable(InterfaceID, PeerEID)}
{
\metP
    {\em Interface ID}: Identifies the interface from which the peer is
    reachable.\\
    {\em Peer EID}: The endpoint ID of the DTN node that can be reached from
    this interface.

\metD
    This is a ``pass-through'' event for the BPA; the BPA is expected to
    pass the message to the DP, and need not take any other action.

    This event message is used only if the CLA has not been configured to
    automatically create a link when a new neighbor is discovered. (If the CLA
    has been configured to automatically create a link, when the link is
    created, reachability is indicated to the DP by the LinkAttributes in the
    EventLinkCreated message.)
    When the CLA is not configured to automatically create
    new links, the EventEIDReachable message is sent to inform the DP that
    neighbor discovery has determined that a new peer EID is reachable, so that
    the DP can determine whether to create a corresponding link or not.

\metR
    RequestAddLink
}

\method{EventLinkCreated(LinkID, LinkAttributes, Reason)}
{
\metD
    See DP interface (section~\ref{sec:DP-EventMessages}).

\metR
    RequestAddLink
}

\method{EventLinkDeleted(LinkID, LinkAttributes, Reason)}
{
\metD
    See DP interface (section~\ref{sec:DP-EventMessages}).

\metR
    RequestDeleteLink
}

\method{EventLinkOpened(LinkID, ContactAttributes)}
{
\metD
    See DP interface (section~\ref{sec:DP-EventMessages}).

\metR
    RequestOpenLink\\
    EventLinkClosed
}

\method{EventLinkClosed(LinkID, ContactAttributes)}
{
\metD
    See DP interface (section~\ref{sec:DP-EventMessages}).

\metR
    RequestCloseLink\\
    EventLinkOpened
}

\method{EventLinkAttributeChange(LinkID, LinkAttributes, Reason)}
{
\metD
    See DP interface (section~\ref{sec:DP-EventMessages}).

    This is a ``pass-through'' event for the BPA; the BPA is expected to
    pass the message to the DP, and need not take any other action.

\metR
    EventContactAttributeChange
}

\method{EventContactAttributeChange(LinkID, ContactAttributes, Reason)}
{
\metD
    See DP interface (section~\ref{sec:DP-EventMessages}).

    This is a ``pass-through'' event for the BPA; the BPA is expected to
    pass the message to the DP, and need not take any other action.

\metR
    EventLinkAttributeChange
}

The following event messages inform the BPA that the CLA has added or changed
the state of a link. In response, the BPA must update any state information it
maintains for links, by either adding a link or changing the state of an
existing one. The CLA has already performed the action so the BPA does not need
to send a request to the CLA to effect the change.\\[1em]

\method{EventLinkStateChange(LinkID, LinkState, Reason)}
{
\metP
    {\em Link ID}: Identifies the link changing state.\\
    {\em Link State}: The state the link has changed to.\\
    {\em Reason}: A reason for the change.

\metD
    When sent by the CLA, this event informs the BPA of a change in the link
    state based on some knowledge available to the CLA. The BPA should
    update its own idea of the link state accordingly. The EventLinkOpened and
    EventLinkClosed messages sent by the CLA set a link's OPEN state. The CLA
    will send the EventLinkStateChange message for other state changes.

    The CLA will set a link to BUSY when the queue has grown too long and will
    return the link to OPEN when bundles can be sent again. A problem with
    this approach must be resolved: a large number of bundle send request
    messages could be sent between the time the CLA requests the link be
    changed to BUSY and the time the BPA receives the message and changes
    the link state. The CLA may not be able to honor these send requests
    received after requesting the change to BUSY.

    Note: This translates to the current LinkStateChangeRequest event in DTN2,
    but the CLA has already changed its state so the BPA should only take
    actions regarding its own link state.

\metR
    EventLinkOpened\\
    EventLinkClosed
}

\method{EventAddReachableLink(LinkID, PeerEID, [LinkConfigParameters])}
{
\metP
    {\em Link ID}: Identifies the link to be created.\\
    {\em Peer EID}: The endpoint ID for the peer of the link being added.\\
    {\em Link Parameters (Opt)}: Link attributes that may be set at link
    creation time, i.e.\ the parameters available for RequestAddLink.

\metD
    If the CLA is configured to create links when discovery determines a new
    neighbor is reachable, the CLA must tell the BPA to create the link, so
    that the BPA can set up the state it keeps for links. This request tells
    the BPA to create an OPPORTUNISTIC link, and set the link to AVAILABLE.
    The BPA does not need to send a RequestAddLink message in response, because
    the CLA has already created the link. The CLA will next send an
    EventLinkCreated message to inform the DP and BPA of the creation and
    reachability of the link.

\metR
    EventLinkCreated\\
    RequestAddLink
}

The following events are directly related to bundles.\\[1em]

\method{EventBundleTransmitted(GBOF-ID, LinkID, TotalBytesSent, TotalBytesSentReliably)}
{
\metP
    % copied from DP
    {\em GBOF ID}: Identifies the bundle or fragment that was transmitted.\\
    {\em Link ID}: Identifies the link over which it was transmitted.\\
    {\em Total bytes sent}: Number of bytes of data (including headers)
    transmitted.\\
    {\em Total bytes sent reliably}: Number of bytes of data (including
    headers) sent reliably. This could be different from total bytes sent
    if there were no ACKs received for some part of the data.

\metD
    This event message is sent when all or part of a bundle has been
    transmitted. If the ``total bytes sent'' is less than the payload plus
    header size of the bundle, the bundle has been reactively fragmented. This
    can happen if the link is closed during transmission of the bundle. A
    link parameter controls whether the CLA will attempt to perform reactive
    fragmentation. If the CLA provides a means for the receiving node to
    acknowledge receipt of data, ``total bytes sent reliably'' will specify how
    many of the ``total bytes sent'' were acknowledged.

    If a bundle has been partially transmitted but reactive fragmentation is
    not performed, EventBundleTransmitFailed will be sent instead. If the
    bundle send is canceled before any portion of it has been sent, the
    EventBundleSendCancelled message is sent instead.

\metR
    EventBundleTransmitFailed\\
    EventBundleSendCancelled\\
    RequestSendBundle
}

\method{EventBundleSendCancelled(GBOF-ID, LinkID)}
{
\metP
    {\em GBOF ID}: Identifies the bundle whose sending was canceled.\\
    {\em Link ID}: Identifies the link for which the bundle send was
    canceled.

\metD
    This message is sent when a bundle that was previously requested to be
    sent has been successfully canceled by a RequestCancelBundleSend
    message.

\metR
    RequestCancelBundleSend
}

\method{EventBundleReceiveStarted(BundleLocation, [PrevHopEID])}
{
\metP
    {\em Bundle Location}: Specifies where the CLA will store the bundle.\\
    {\em Previous Hop EID}: The node from which this bundle is being
    received, when available.

\metD
    This message is sent by the CLA when it begins to receive a bundle. If
    the CLA provides a means of acknowledging receipt of data to the peer
    (such as in the TCP CL), this message must be sent before the CLA
    acknowledges any of the bundle data. If there is no means of
    acknowledging receipt of data (such as with the UDP CL), the CLA is not
    required to send this message to the BPA.

    The motivation for this message is to handle cases where the external
    CLA crashes or is somehow unable to send messages to the BPA between
    acknowledging data to the peer and sending the EventBundleReceived
    message to the BPA. If the CLA acks data and then cannot post the
    BundleReceivedEvent, the peer transmitting the bundle may assume the
    bundle is transmitted (or perform reactive fragmentation), yet the
    bundle has not actually arrived at the BPA. This could result in the
    bundle or fragment being lost.

    If the BPA finds that the CLA has died, it should collect and process
    all bundles for which EventBundleReceiveStarted was sent but not
    EventBundleReceived.

\metR
    EventBundleReceived
}

\method{EventBundleReceived(BundleLocation, BytesReceived, [PrevHopEID])}
{
\metP
    {\em Bundle Location}: Specifies where the CLA stored the bundle.\\
    {\em Bytes Received}: Total number of bytes in the bundle (including
    headers).\\
    {\em Previous Hop EID (Opt)}: The node from which this bundle was 
    received, when available.

\metD
    This message is sent when the CLA has completed receiving a bundle. If the
    ``bytes received'' is less than the bundle length, then the bundle has been
    reactively fragmented. This can happen when the link is closed during
    reception of the bundle. A link parameter controls whether the CLA will
    attempt to perform reactive fragmentation. 

\metR
    EventBundleReceiveStarted
}

\subsubsection{CLA Registration}

These two requests are used by the CLA at startup and shutdown.
They are sent from the CLA to the BPA. After connecting to the BPA, the CLA
sends a RequestAddCLA message to inform the BPA that the CLA has been created
and is available to handle future requests from the BPA. If the CLA is shutting
down, it informs the BPA of this fact with a RequestDeleteCLA message.\\[1em]

\method{RequestAddCLA(CLAName)}
{
\metP
    {\em CLA Name}: Identifies the CLA that is requesting to be added.

\metD
    While starting up, the CLA should send this request after connecting to the
    BPA. This informs the BPA that the CLA is ready for future requests.

\metR
    RequestDeleteCLA
}

\method{RequestDeleteCLA(CLAName)}
{
\metP
    {\em CLA Name}: Identifies the CLA that is being deleted.

\metD
    When shutting down the CLA sends this request message to inform the BPA
    that the CLA is shutting down. The BPA is then expected to take correct
    actions for a CLA going away.

    It is possible for the CLA to die without sending this request, so the BPA
    should detect this situation and take the correct actions for deleting a
    CLA.

\metR
    RequestAddCLA
}

\subsubsection{Query/Report Messages}

Query messages are sent to the CLA by the BPA. Some queries may originate at
the DP and simply be passed to the CLA by the BPA. The corresponding Report
messages are sent from the CLA to the BPA.\\[1em]

\method{QueryBundleQueued(QueryID, GBOF-ID, LinkID)}
{
\metP
    {\em QueryID}: Identifies the query (a handle) to tie the report.\\
    {\em GBOF-ID}: Identifies the bundle being queried.\\
    {\em Link ID}: Identifies the link being queried.

\metD
    This query is for determining whether or not a bundle is currently on
    the send queue of the given link. A unique query id is passed and a
    report message containing the query id is awaited. The report contains
    the requested information.

\metR
    ReportBundleQueued
}

\method{ReportBundleQueued(QueryID, YesOrNo)}
{
\metP
    {\em QueryID}: Identifies which query is being responded to.\\
    {\em YesOrNo}: Yes if the bundle is on the link's send queue.

\metD
    This message is a response to a query and reports whether or not the
    specified bundle is currently queued to be transmitted on the specified
    link. The query id ties the report to the query. 

\metR
    QueryBundleQueued
}

\method{QueryEIDReachable(QueryID, InterfaceID, PeerEID)}
{
\metP
    {\em QueryID}: Identifies the query (a handle) to tie the report.\\
    {\em Interface ID}: Identifies the interface being queried.\\
    {\em Peer EID}: Identifies the endpoint ID being queried.

\metD
    This query is for determining whether or not an endpoint ID is reachable
    from the given interface. A unique query id is passed and a
    report message containing the query id is awaited. The report contains
    the requested information.

\metR
    ReportEIDReachable
}

\method{ReportEIDReachable(QueryID, YesOrNo)}
{
\metP
    {\em QueryID}: Identifies which query is being responded to.\\
    {\em YesOrNo}: Yes if the peer endpoint ID is reachable from the queried
    interface.

\metD
    This message is a response to a query and reports whether or not the
    specified endpoint ID is currently reachable from the specified interface.
    The query id ties the report to the query. 

\metR
    QueryEIDReachable
}

\method{QueryLinkAttributes(QueryID, LinkID, QueryParams)}
{
\metD
    See DP interface (section~\ref{sec:DP-QueryReportMessages}).
%   COMMENT: This would query measured link characteristics. Presumably there
%   is a well-defined set of characteristics such as bandwidth and RTT,
%   with additional CLA-specific characteristics available. This could be
%   allowed to query the parameters set at link creation time as well. It
%   could also query the link state (open or closed).
    
\metR
    ReportLinkAttributes
}

\method{ReportLinkAttributes(QueryID, ReportParams)}
{
\metD
    See DP interface (section~\ref{sec:DP-QueryReportMessages}).

\metR
    QueryLinkAttributes
}

\method{QueryInterfaceAttributes(QueryID, InterfaceID, QueryParams)}
{
\metP
    % copied from DP
    {\em QueryID}: Identifies the query (a handle) to tie the report.\\
    {\em Interface ID}: Identifies the interface being queried.\\
    {\em QueryParams}: A list of keys. The key identifies which information
    object is asked for. A list of key types is defined somewhere.

\metD
    % copied from DP
    This query is for getting the value of any subset of a number of
    pre-defined interface attributes. A unique query id is passed and a
    report message containing the query id is awaited. The report contains
    the requested information (key,value) pairs (see below).

\metR
    ReportInterfaceAttributes
}

\method{ReportInterfaceAttributes(QueryID, ReportParams)}
{
\metP
    % copied from DP
    {\em QueryID}: Identifies which query is being responded to.\\
    {\em ReportParams}: A list of (key, value) pairs.
    
\metD
    % copied from DP
    This message is a response to a query and reports the latest value
    of the interface attributes that were queried for. The query id ties
    the report to the query. The ReportParams contains (key, value)
    pairs, where each key is taken from the list of keys supplied with
    the corresponding query, and the value is the value of whatever is
    requested for by the key.

\metR
    QueryInterfaceAttributes
}

\method{QueryCLAParameters(QueryID, QueryParams)}
{
\metP
    % copied from DP
    {\em QueryID}: Identifies the query (a handle) to tie the report.\\
    {\em Interface ID}: Identifies the interface being queried.\\
    {\em QueryParams}: A list of keys. The key identifies which information
    object is asked for. A list of key types is defined somewhere.
    
\metD
    % copied from DP
    This query is for getting the value of any subset of a number of
    pre-defined CLA parameters. A unique query id is passed and a
    report message containing the query id is awaited. The report contains
    the requested information (key,value) pairs (see below).

\metR
    ReportCLAParameters
}

\method{ReportCLAParameters(QueryID, ReportParams)}
{
\metP
    % copied from DP
    {\em QueryID}: Identifies which query is being responded to.\\
    {\em ReportParams}: A list of (key, value) pairs.
    
\metD
    % copied from DP
    This message is a response to a query and reports the latest value
    of the CLA parameters that were queried for. The query id ties
    the report to the query. The ReportParams contains (key, value)
    pairs, where each key is taken from the list of keys supplied with
    the corresponding query, and the value is the value of whatever is
    requested for by the key.

\metR
    QueryCLAParameters
}
